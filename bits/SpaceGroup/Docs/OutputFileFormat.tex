\documentclass[11pt]{article}
\usepackage{graphicx}
\usepackage{amssymb}
\usepackage{xspace}
\usepackage{epstopdf}
\DeclareGraphicsRule{.tif}{png}{.png}{`convert #1 `basename #1 .tif`.png}

\textwidth = 6.5 in
\textheight = 9 in
\oddsidemargin = 0.0 in
\evensidemargin = 0.0 in
\topmargin = 0.0 in
\headheight = 0.0 in
\headsep = 0.0 in
\parskip = 0.2in
\parindent = 0.0in

\newcommand{\til}{\char '176}  %The tilde char
\newcommand{\ands}{\char '46} %The and char
\newcommand{\lar}{\char '74}	%left arrow
\newcommand{\rar}{\char '46}  %right arrow
\newcommand{\hash}{\char '43} %hash char
\newtheorem{theorem}{Theorem}
\newtheorem{corollary}[theorem]{Corollary}
\newtheorem{definition}{Definition}

\title{Output file format}
\author{Stefan Pantos}
\begin{document}
\maketitle
\section{Introduction}
The document is to describe the format of the file which is outputted from the program TheDeterminator. This is to give a referance to any other developer who would like to use this file. The format was origanly designed for intergation to the Crystallographic program Crystals.\\\\
\section{File format}
\small{\hash in integer number. \hash . \hash a floating point number.}
\subsection{High level view}
The file is spit into sections these sections are:\\
\begin{table}[h]
\begin{tabular}{|l|}\hline
Header\\
\hline
Regions Used\\
\hline
Conditions Used\\
\hline
Stats Table\\
\hline
Results\\
\hline
\end{tabular}
\end{table}
\subsection{Header}
\begin{table}[h]
\begin{tabular}{|l|l|}\hline
TOTAL \hash & \hash  is the number of reflections which where\\
		&	read in from the .hkl file.\\
\hline
AVINT \hash . \hash & \hash . \hash average intensity of all the reflections in the .hkl file.\\
\hline
\end{tabular}
\end{table}
e.g.\\
TOTAL 13390\\
AVINT 67.0403
\subsection{Regions Used}
\begin{table}[h]
\begin{tabular}{|l|l|}\hline
NREGIONS \hash & \hash is the number of regions which where used in the calculation\\
		&	and the number of regions which are to follow.\\
\hline
\hash index Matrix Region & \hash index is the index of the region as is in the table file.\\
					& Matrix is the 3x3 matrix which defines the region. It is in the form\\
					& \hash  \hash  \hash  \hash  \hash  \hash  \hash  \hash  \hash.\\
					& Region is the text representation of the region. eg. h0l\\
					& \emph{There is a line like this for each region.}\\
\hline
\end{tabular}
\end{table}
e.g.\\
NREGIONS 3\\
0 1 0 0 0 1 0 0 0 1 hkl\\
1 0 0 0 0 1 0 0 0 1 0kl\\
6 0 0 0 0 0 0 0 0 1 00l
\subsection{Condtions Used}
\begin{table}[h]
\begin{tabular}{|l|l|}\hline
NTEST \hash & \hash is the number of conditions which where used in the calculation\\
		&	and the number of conditons which are to follow.\\
\hline
\hash index Matrix Multi Condition & \hash index is the index of the condition as is in the table file.\\
					& Matrix is the 1x3 matrix which defines the condtion. It is in the form\\
					& \hash  \hash  \hash.\\
					& Multi is the multiplier used in the condition. This has the form \\hash.\\
					& Condtion is the text representation of the condition. eg. h == 2n\\
					& \emph{There is a line like this for each region.}\\
\hline
\end{tabular}
\end{table}
e.g.\\
NTESTS 3
0 1 0 0 2 h == 2n
1 0 1 0 2 k == 2n
8 0 0 1 4 l == 4n

\subsection{Stats Table}
\begin{table}[h]
\begin{tabular}{|l|l|}\hline
DATA \hash  \hash & First number is the number or sets of stats N. The second number is the number of stats give. Currently this is set to 7.\\
\hline
N sets of Stats data &\\\hline
\end{tabular}
\end{table}
\subsubsection{Stats data}
\begin{table}[h]
\begin{tabular}{|l|l|}\hline
\hash & The number of reflections which matched the condition.\\\hline
\hash.\hash & The average intensity of the reflections which match the conditon.\\\hline
\hash & The number of reflections which didn't matched the condition.\\\hline
\hash.\hash & The average intensity of the reflections which didn't match the conditon.\\\hline
\hash.\hash & Percentage of reflections didn't match which where I < 3*u(I).\\\hline
\hash.\hash & Score 1 also called ratio 1. See math's reference for further details.\\\hline
\hash.\hash & Score 2 also called ratio 1. See math's reference for further details.\\\hline
\hline
\end{tabular}
\end{table}
e.g.\\
DATA 21 7\\
6710\\
78.0809\\
6680\\
55.9501\\
38.8174\\
-0.31918\\
-0.45224\\
6705\\
78.17\\
6685\\
55.8772\\
38.5639\\
-0.31918\\
-0.4562\\
6702\\
78.2019\\
6688\\
55.8553\\
38.6812\\
-0.31561\\
-0.4562\\
 \end{document}